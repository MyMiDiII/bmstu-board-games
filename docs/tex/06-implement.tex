\chapter{Технологическая часть}

\section{Выбор системы управления базами данных}

Система управления базами данных (СУБД) --- это совокупность программ и языковых
средств, предназначенных для управления данными в базе данных, ведения базы
данных и обеспечения взаимодействия ее с прикладными программами~\cite{gost01}.

В предыдущем подразделе для реализуемой системы были выбраны реляционные
базы данных, следовательно, необходимо рассмотреть СУБД, предоставляющие
возможность работы с ними:
\begin{itemize}
    \item Microsoft SQL Server,
    \item PostgreSQL,
    \item Oracle,
    \item MySQL.
\end{itemize}

\subsection{Microsoft SQL Server}

Microsoft SQL Server --- реляционная система управления базами данных,
разработанная корпорацией Microsoft, основным языком запросов
которой является Transact-SQL~\cite{art06}.

Достоинствами данной СУБД являются:
\begin{itemize}
    \item устойчивая производительность;
    \item возможность работы с облачными хранилищами;
    \item возможность интеграции с другими продуктами Microsoft.
\end{itemize}

Недостатками данной СУБД являются:
\begin{itemize}
    \item высокая стоимость корпоративной версии;
    \item поддержка меньшего числа операционных систем (только Windows
        и Linux) по сравнению с другими реляционными СУБД~\cite{art07}.
\end{itemize}

\subsection{PostgreSQL}

PostgreSQL --- это объектно-реляционная система управления базами данных с
открытым исходным кодом, поддерживающая соответствие стандартам ANSI
SQL~\cite{site08}.

Достоинствами данной СУБД являются:
\begin{itemize}
    \item открытый исходный код и свободная лицензия;
    \item полное соответствие требованиям ACID;
    \item поддержка пользовательских типов данных, в том числе
        неструктурированных (JSON и XML);
    \item поддержка большого количества сторонних инструментов;
    \item кроссплатформенность.
\end{itemize}

Недостатками данной СУБД являются:
\begin{itemize}
    \item более низкая скорость по сравнению с другими СУБД.
\end{itemize}

\subsection{Oracle}

Oracle --- это объектно-реляционная система управления базами данных, созданная
корпорацией Oracle~\cite{site09}.

Достоинствами данной СУБД являются:
\begin{itemize}
    \item поддержка работы с крупными БД и большим числом пользователей;
    \item высокая производительность;
    \item сильная техническая поддержка, подробная документация;
    \item кроссплатформенность.
\end{itemize}

Недостатками данной СУБД являются:
\begin{itemize}
    \item высокая стоимость;
    \item ресурсоемкость.
\end{itemize}

\subsection{MySQL}

MySQL --- это бесплатная реляционная система управления базами данных,
разработку и поддержку которой осуществляет корпорация Oracle~\cite{art06}.

Достоинствами данной СУБД являются:
\begin{itemize}
    \item бесплатное распространение;
    \item масштабируемость и производительность;
    \item возможность работы с облачными хранилищами;
    \item кроссплатформенность.
\end{itemize}

Недостатками данной СУБД являются:
\begin{itemize}
    \item платная поддержка;
    \item меньшая надежность по сравнению с другими СУБД.
\end{itemize}

\subsection*{Вывод}

Для решения задачи была выбрана система управления базами данных
PostgreSQL, так как она:

\begin{itemize}
    \item поддерживает полное соответствие требованиям ACID, что
позволит сохранять целостность данных в том числе при параллельной работе
нескольких пользователей с системой;
    \item имеет открытый исходный код;
    \item является гибкой в работе с типами данных и расширениями.
\end{itemize}

, а в качестве СУБД для работы с ней --- PostgreSQL

\section{Выбор средств реализации приложения}

\section{Детали реализации}

\subsection{Создание таблиц}

{
\captionsetup{format=hang,justification=raggedright,
              singlelinecheck=off,width=16cm}
\listingfile{sql}{createTables.sql}{firstline=15,lastline=38}{Создание таблицы игротек}{events}

\listingfile{sql}{createTables.sql}{firstline=54,lastline=67}{Создание
таблицы игр}{games}

\listingfile{sql}{createTables.sql}{firstline=123,lastline=133}{Создание
таблицы мест проведения}{venues}

\listingfile{sql}{createTables.sql}{firstline=69,lastline=78}{Создание
таблицы организаторов}{orgs}

\listingfile{sql}{createTables.sql}{firstline=80,lastline=87}{Создание
таблицы игроков}{players}

\listingfile{sql}{createTables.sql}{firstline=116,lastline=121}{Создание
таблицы пользователей}{users}

\listingfile{sql}{createTables.sql}{firstline=103,lastline=113}{Создание
таблицы ролей}{roles}

\listingfile{sql}{createTables.sql}{firstline=1,lastline=13}{Создание
таблицы связи игр и игротек}{egrs}

\listingfile{sql}{createTables.sql}{firstline=40,lastline=52}{Создание
таблицы связи игр и игроков}{favs}

\listingfile{sql}{createTables.sql}{firstline=89,lastline=101}{Создание
таблицы связи игротек и игроков}{regs}
}

\subsection{Создание ролей}

\subsection{Создание хранимых процедур и функций}

\subsection{Создание триггеров}

\section{Пример работы программы}
