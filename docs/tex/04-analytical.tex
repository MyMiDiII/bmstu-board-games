\chapter{Аналитическая часть}

\section{Анализ предметной области}

Игротека проводится организатором (магазином) в каком-то месте (кафе, антикафе,
ресторане и тп). На ней представлена одна или несколько игр, в которые можно
поиграть. Часто организаторы предоставляют возможность покупки тех же игр прямо
на месте или возможность сделать заказ.

\section{Постановка задачи}

Задача.

\section{Формализация данных}

Из анализа предметной области информация о сущностях.

\section{Описание типов пользователей}

Админ

Организатор

Игрок

\section{Анализ существующих решений}

\subsection{В России}

В сфере настольных игр есть несколько компаний (Hobby World, МосИгра,
NizaGames), представляющие свои интернет магазины настольных игр. Веб-сайты
ориентированы именно на продажу игр, в то время как для анонсы игротек
второстепенны и организатор соответствует организации сайта, появляются на одной
странице или в новостях. То есть решений с возможностью просмотра многих игротек
разных организаторов в <<одном месте>> на российском рынке найдено не было

\subsection{Иностранные}

BoardGameGeek --- многофункциональный сайт для любителей настольных игр,
содержит информацию о многочисленных настольных играх, предоставляя ссылки на
покупку, также предоставляет форумы, топы и т д и тп. Самое главное содежит
страницу с акутальными данными по игровым конференциям всего мира, на которых
проводятся игротеки.

Toptiergaming --- информация по игротекам США и релизам настольных игр.

Board Game Halv --- новостной сайт о мире настолок. Содержит страничку в
игротеками, но больше

\subsection{Вывод}

Полноценных аналогов не найдено.

\section{Модели БД и СУБД}

\subsection{Базы данных}

\subsection{Системы управления базами данных}

\subsection{Вывод}

Выбираю PostgreSQL, потому что я умею с ним работать :)

\section{Вывод}
