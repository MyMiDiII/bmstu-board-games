\chapter{Аналитическая часть}

\section{Анализ предметной области}

Игротека проводится организатором (магазином) в каком-то месте (кафе, антикафе,
ресторане и тп). На ней представлена одна или несколько игр, в которые можно
поиграть. Часто организаторы предоставляют возможность покупки тех же игр прямо
на месте или возможность сделать заказ.

\textbf{\texttt{Анализ}} страниц сайтах об игротеках показал, что
\textbf{\texttt{обычно}} информация об игротеке включает дату, время начала,
продолжительность или время окончания, адрес проведения и стоимость участия.
Также представляется набор игр, по которым проводится игротека.

Информация об играх 

\section{Постановка задачи}

Задача.

\section{Формализация данных}

На основе анализа предметной области, ..., база данных должна содержать
информацию о:

\begin{itemize}
    \item игротеках;
    \item организаторах;
    \item местах проведения;
    \item настольных играх, по которым проводятся игротеки;
    \item игроках;
    \item пользователях.
\end{itemize}

Сведения о каждой представленной сущности представлены в таблице~\ref{tab:01}.

\textbf{\texttt{Типы здесь?}}

\begin{table}[h!]
    \begin{center}
    \begin{threeparttable}
        \captionsetup{format=hang,justification=raggedright,
                      singlelinecheck=off}
        \caption{\label{tab:01}Категории данных и сведения о них}
        \renewcommand{\arraystretch}{1.5}
        \begin{tabular}{|p{4cm}|p{12cm}|}
            \hline
            \bfseries Категория & \bfseries Сведения\\
            \hline
            Игротека & ID\par
                       Название\par
                       Дата проведения\par
                       Время начала\par
                       Продолжительность\par
                       Стоимость участия\par
                       Возможность покупки игр\par
                       Организатор\par
                       Место проведения\\
            \hline
            \begin{minipage}[t]{\linewidth}
                              \begin{multicols}{1}
                                  \begin{itemize}[labelsep=0mm,
                                                  nosep,after=\strut]
                                      \item[] Организатор
                                      \item[]
                                      \item[]
                                      \item[]
                                      \item[]
                                  \end{itemize}
                              \end{multicols}
                          \end{minipage}
                     & \begin{minipage}[t]{\linewidth}
                              \begin{multicols}{2}
                                  \begin{itemize}[labelsep=0mm,
                                                  nosep,after=\strut]
                                      \item[] ID
                                      \item[] Название
                                      \item[] Адрес
                                      \item[] Сайт
                                      \item[] Email
                                      \item[] Телефон
                                  \end{itemize}
                              \end{multicols}
                          \end{minipage}\\
            \hline
            Место проведения & ID\par
                               Название\par
                               Тип (магазин, кафе, антикафе и т.~п.)\par
                               Адрес\par
                               Сайт\par
                               Email\par
                               Телефон\\
            \hline
            Настольная игра & ID\par
                              Название\par
                              Производитель\par
                              Год выпуска\par
                              Минимальный возраст\par
                              Максимальный возраст\par
                              Минимальное число игроков\par
                              Максимальное число игроков\par
                              Минимальное время игры\par
                              Максимальное время игры\\
            \hline
            Игрок & ID\par
                    Имя\par
                    Рейтинг\par
                    Лига\\
            \hline
            Пользователь & ID\par
                           Логин\par
                           Пароль\par
                           Роль\\
           \hline
        \end{tabular}
    \end{threeparttable}
    \end{center}
\end{table} 

\clearpage
\section{Описание типов пользователей}

\textbf{\texttt{Продумать просмотр, чего нужен каждому}}

\begin{table}[h!]
    \begin{center}
    \begin{threeparttable}
        \captionsetup{format=hang,justification=raggedright,
                      singlelinecheck=off}
        \caption{\label{tab:02}Описание типов пользователей}
        \renewcommand{\arraystretch}{1.5}
        \begin{tabular}{|p{3.3cm}|p{5.2cm}|p{7cm}|}
            \hline
            \bfseries Тип & \bfseries Описание &
            \bfseries Функциональность\\
            \hline
            Гость & Неавторизированный\par пользователь
                  & \begin{minipage}[t]{\linewidth}
                      \begin{itemize}[nosep,after=\strut]
                        \item Просмотр списка настольных игр и игротек
                        \item Авторизация
                        \item Регистрация
                      \end{itemize}
                  \end{minipage}\\
            \hline
            Игрок & Авторизированный пользователь
                  & \begin{minipage}[t]{\linewidth}
                      \begin{itemize}[nosep,after=\strut]
                          \item Просмотр списка настольных игр и игротек
                          \item Составление списка предпочитаемых игр
                          \item Регистрация на игротеки
                      \end{itemize}
                  \end{minipage}\\
            \hline
            Организатор & Авторизированный пользователь\par
                          с возможностью работы с игротеками
                  & \begin{minipage}[t]{\linewidth}
                      \begin{itemize}[nosep,after=\strut]
                          \item Просмотр списка настольных игр, игротек, мест
                        проведения, организаторов\
                          \item Регистрация новых игротек
                          \item Просмотр списка зарегистрированных на игротеку
                        участников
                          \item Формирование заявки на добавление нового места
                        проведения при его отсутствии в базе
                          \item Формирование заявки на добавление игры при ее
                        отсутствии в базе
                      \end{itemize}
                  \end{minipage}\\
            \hline
            Администратор & Авторизированный пользователь\par
                            с повышенным уровнем полномочий
                  & \begin{minipage}[t]{\linewidth}
                      \begin{itemize}[nosep,after=\strut]
                          \item Просмотр списка настольных игр, игротек, мест
                        проведения, организаторов, пользователей
                          \item Изменение информации о настольных играх, местах
                        проведения, организаторах, игротеках
                          \item Одобрение или отклонение заявок от
                        организаторов
                          \item Изменение прав доступа пользователей
                          \item Удаление пользователей
                      \end{itemize}
                  \end{minipage}\\
            \hline
        \end{tabular}
    \end{threeparttable}
    \end{center}
\end{table} 


\section{Анализ существующих решений}

\subsection{В России}

В сфере настольных игр есть несколько компаний (Hobby World, МосИгра,
NizaGames), представляющие свои интернет магазины настольных игр. Веб-сайты
ориентированы именно на продажу игр, в то время как для анонсы игротек
второстепенны и организатор соответствует организации сайта, появляются на одной
странице или в новостях. То есть решений с возможностью просмотра многих игротек
разных организаторов в <<одном месте>> на российском рынке найдено не было

\subsection{Иностранные}

BoardGameGeek --- многофункциональный сайт для любителей настольных игр,
содержит информацию о многочисленных настольных играх, предоставляя ссылки на
покупку, также предоставляет форумы, топы и т д и тп. Самое главное содежит
страницу с акутальными данными по игровым конференциям всего мира, на которых
проводятся игротеки.

Toptiergaming --- информация по игротекам США и релизам настольных игр.

Board Game Halv --- новостной сайт о мире настолок. Содержит страничку в
игротеками, но больше

\subsection{Вывод}

Полноценных аналогов не найдено.

\section{Модели БД и СУБД}

\subsection{Базы данных}

\subsection{Системы управления базами данных}

\subsection{Вывод}

Выбираю PostgreSQL, потому что я умею с ним работать :)

\section{Вывод}
