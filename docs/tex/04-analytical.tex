\chapter{\label{analys}Аналитическая часть}

\begin{comment}
\end{comment}

В данном разделе проводится анализ предметной области: на основе рассмотрения
существующих решений делаются выводы о представлении информации о настольных
играх и игротеках, а также о необходимых возможностях пользователей. На базе
анализа предметной области формулируются требования к информационной системе,
формализуются данные и типы пользователей. Также проводится сравнение
существующих моделей баз данных и систем управления базами данных, из которых
выбираются оптимальные для решения поставленной задачи.

\section{Анализ существующих аналогов}

В силу молодости рынка настольных игр анализ предметной области проводится на
основе рассмотрения программных решений, представляющих интернет-магазины
крупных компаний настольных игр или тематических сайтов.

Так как рынки настольных игр в России и за рубежом имеют разные уровни развития
существующие программные решения рассматриваются отдельно для российского и
зарубежного рынков.

\subsection{Российский рынок}

На российском рынке программные решения в сфере настольных игр представлены
интернет-магазинами крупных продавцов: Hobby Games~\cite{site01},
<<Мосигра>>~\cite{site02}, <<Низа Гамс>>~\cite{site03} и др. Их веб-сайты
ориентированы именно на продажу игр, в то время как анонсы игротек являются
второстепенными и располагаются либо в специализированном разделе сайта, либо в
новостях, при этом регистрация проходит либо через сторонний сервис, либо через
сообщения в социальных сетях.

\subsection{Зарубежный рынок}

На зарубежном рынке преобладают тематические сайты. Например,
BoardGameGeek~\cite{site04} --- многофункциональный сайт для любителей
настольных игр, --- содержит информацию о многочисленных настольных играх, их
авторах, предоставляет ссылки на интернет-магазины, в которых можно купить игру,
а также возможность продать свою игру и пообщаться на форуме. Данный сайт также
содержит страницу с актуальными данными по игровым конференциям всего мира, на
которых проводятся игротеки. Аналогичными проектами, но с меньшими наборами
функций являются Top Tier Board Games~\cite{site05} и Board Game
Halv~\cite{site06}, они представляют информацию о настольных играх и игротеках в
формате новостей.

Существует также программное решение, предоставляющее возможность играть в
популярные настольные игры онлайн --- Board Game Arena~\cite{site07}. Здесь
реализованы виртуальные копии многих настольных игр, ведутся рейтинги по каждой
игре и по всем играм в целом. Благодаря возможности перевода сайта
пользователями, он доступен на нескольких языках, в том числе и на русском.

\subsection{Сравнение существующих решений}

Для сравнения существующих решений можно выделить следующие критерии:
\begin{itemize}
    \item \textbf{К1} --- наличие информации о настольных играх;
    \item \textbf{К2} --- наличие информации об игротеках;
    \item \textbf{К3} --- наличие информации об игротеках различных
        организаторов;
    \item \textbf{К4} --- возможность регистрации на игротеку;
    \item \textbf{К5} --- ведение рейтингов игроков;
    \item \textbf{К6} --- поддержка русского языка.
\end{itemize}

Сравнение решений по данным критериям представлено в таблице~\ref{tab:01}.

\begin{table}[ht!]
\captionsetup{format=hang,justification=raggedright,
              singlelinecheck=off,width=13.5cm}
\centering
\caption{Сравнение существующих решений\label{tab:01}}
\begin{tabular}[Hc]{|p{4cm}|c|c|c|p{3.5cm}|c|c|}
    \hline
    \multicolumn{1}{|c}{\textbf{Решение}} & \multicolumn{1}{|c|}{\textbf{K1}} &
    \multicolumn{1}{c|}{\textbf{K2}} & \multicolumn{1}{c}{\textbf{K3}} &
    \multicolumn{1}{|c|}{\textbf{К4}} & \multicolumn{1}{c|}{\textbf{K5}} &
    \multicolumn{1}{c|}{\textbf{К6}}\\
    \hline
    \mbox{Hobby Games}      & + & + & - & через сторонний ресурс & - & + \\
    \hline                                                           
    \mbox{Мосигра}          & + & - & - & нет                    & - & + \\
    \hline                                                           
    \mbox{Низа Гамс}        & + & + & - & через сторонний ресурс & - & + \\
    \hline                                                           
    \mbox{BoardGameGeek}    & + & + & + & через сторонний ресурс & - & - \\
    \hline                                                           
    \mbox{Board Game Arena} & + & - & - & нет                    & + & + \\
    \hline
\end{tabular}
\end{table}

\clearpage
Данные из приведенной таблицы показывают, что все описанные существующие решения
предоставляют пользователю возможность просмотра информации о настольных играх,
однако при этом возможноть работы с игротеками в них либо вовсе отсутствует,
либо предоставляется через сторонний ресурс. Также стоит отметить, что только
одно решения предоставляет возможность работы с игротеками разных организаторов
и только одно возможность отслеживания рейтинга игрока.

\subsection*{Вывод}

Таким образом, программные решения для поиска информации о настольных играх
широко распространены, однако решение задачи поиска необходимой игротеки для
игрока требует многих усилий. Если организаторы игротек, используя свои сайты,
имеют возможность в любом удобном для них формате выкладывать информацию об
игротеках, то для поиска нужной игротеки игроку придется посетить многочисленные
сайты организаторов, найти раздел или страницу игротек на каждом из них и
зарегистрироваться через сторонний ресурс.

То есть встает необходимость реализации информационной системы, в которой
информация о всех игротеках будет собрана в одном месте, а регистрация будет
происходить единообразно в самой системе. При этом в данной системе должна
содержатся информация о настольных играх, так как на их основе происходит выбор
игротеки. А в силу того, что реализация регистрации на игротеки, потребует
регистрации в системе, как организаторов, так и игроков, появится возможность
включения в систему рейтинговых списков.

\section{Формализация задачи}

Для создания информационной системы необходимо формализовать процесс проведения
игротеки. Для этого выделим необходимые и возможные действия пользователей,
которые будут соответствовать требованиям, предъявляемым к
создаваемому программному обеспечению.

И так, при проведении игротеки сначала организатор:
\begin{enumerate}[label=\arabic*)]
    \item выбирает время и место проведения игротеки, а также настольные игры,
      по которым она проводится;
    \item определяет другие характеристики игротеки (название,
        продолжительность, стоимость участия, возможность купить игры);
    \item регистрирует игротеку.
\end{enumerate}

Далее игрок --- участник игротеки:
\begin{enumerate}[label=\arabic*)]
    \item просматривает список настольных игр;
    \item выбирает понравившиеся игры, добавляя их в свой список избранных;
    \item выбирает игротеку по понравившимся играм или из всего списка
        зарегистрированных игротек;
    \item регистрируется на игротеку.
\end{enumerate}

При этом игрок может пропустить этапы просмотра и выбора настольных игр и
сразу перейти к выбору игротеки.

Описанные выше действия субъектов игротеки соответствуют необходимым функциям,
которые являются обязательными требованиями к разрабатываемой информационной
системе и должны быть реализованы в первую очередь. Также различия в возможных
действиях пользователей формируют еще одно обязательное требование: регистрацию
различных типов пользователей в системе.

Также у организатора должна быть возможность:
\begin{itemize}
    \item просмотреть список участников, зарегистрированных на игротеку;
    \item отменить игротеку;
    \item перенести игротеку (изменить место проведения или время начала);
    \item изменить другую информацию о ней;
    \item сформировать заявку на добавление в базу места проведения или
        настольной игры при их отсутствии.
\end{itemize}

Игрок, в свою очередь, должен иметь возможность:
\begin{itemize}
    \item удалить игру из списка понравившихся игр;
    \item отменить регистрацию на игротеку.
\end{itemize}

Данные требования являются желательными и реализуются сразу после обязательных.
 
Возможными требования к реализуемой системе являются:
\begin{itemize}
    \item подтверждение организатором посещения игроком игротеки;
    \item изменение организатором рейтинга игрока после посещения им игротеки;
    \item уменьшение рейтинга игрока вследствие пропуска игротеки, на которую он
        был зарегистрирован;
    \item оценка игроком посещенной игротеки и организатора.
\end{itemize}

Также необходимо учесть ряд правил делового регламента:
\begin{itemize}
    \item организатор не может зарегистрировать игротеку, дата и время начала
        которой прошли или соответствуют текущему дню;
    \item игротека считается прошедшей, если наступила дата проведения и время
        ее окончания;
    \item игрок не может зарегистрироваться на прошедшую или отмененную
        игротеку;
    \item игрок может зарегистрироваться на игротеку не позднее, чем за
        установленное организатором время до ее начала;
    \item при отмене игротеки список зарегистрированных на нее участников
        сохраняется;
    \item регистрация игротеки считается незавершенной до тех пор, пока не будут
        приняты заявки организатора на добавление в базу места проведения и хотя
        бы одной игры, если такие оформлялись.
\end{itemize}

Таким образом, необходимо разработать приложение, отвечающее выше перечисленным
требованиям. В силу необходимости хранения информации о различных объектах
предметной области также требуется спроектировать и реализовать базу данных,
которая должна хранить информацию о настольных играх и игротеках, и
взаимодействие с которой должно осуществляться с помощью интерфейса
разрабатываемого приложения.

\section{Описание типов пользователей}

В соответствии с поставленными требованиями необходимо выделить 4~типа
пользователей, описание которых представлено в таблице~\ref{tab:02}.

\clearpage
\begin{table}[h!]
    \begin{center}
    \begin{threeparttable}
        \captionsetup{format=hang,justification=raggedright,
                      singlelinecheck=off}
        \vspace{-0.7cm}
        \caption{\label{tab:02}Описание типов пользователей}
        \renewcommand{\arraystretch}{1.5}
        \begin{tabular}{|p{5.1cm}|p{11.3cm}|}
            \hline
            \bfseries Тип &
            \bfseries Функциональность\\
            \hline
            Гость \par (неавторизованный пользователь)
                  & \begin{minipage}[t]{\linewidth}
                      \begin{itemize}[nosep,after=\strut]
                        \item Просмотр списка настольных игр и игротек
                        \item Поиск игротек по настольной игре
                        \item Авторизация
                        \item Регистрация
                      \end{itemize}
                  \end{minipage}\\
            \hline
            Игрок \par (авторизованный пользователь)
                  & \begin{minipage}[t]{\linewidth}
                      \begin{itemize}[nosep,after=\strut]
                          \item Просмотр списка настольных игр и игротек
                          \item Составление списка предпочитаемых игр
                          \item Поиск игротек по настольным играм 
                          \item Регистрация на игротеки и ее отмена
                      \end{itemize}
                  \end{minipage}\\
            \hline
            Организатор \par (авторизованный пользователь)
                  & \begin{minipage}[t]{\linewidth}
                      \begin{itemize}[nosep,after=\strut]
                          \item Просмотр списка настольных игр, игротек, мест
                        проведения
                          \item Регистрация новых игротек и их отмена
                          \item Изменение информации о зарегистрированных
                              игротеках
                          \item Просмотр списка зарегистрированных на
                              \mbox{игротеку} участников
                          \item Формирование заявки на добавление нового
                              \mbox{места} проведения при его отсутствии в базе
                          \item Формирование заявки на добавление игры при ее
                        отсутствии в базе
                      \end{itemize}
                  \end{minipage}\\
            \hline
            Администратор \par (авторизованный пользователь
                    с \mbox{повышенным~~уровнем} полномочий)
                  & \begin{minipage}[t]{\linewidth}
                      \begin{itemize}[nosep,after=\strut]
                          \item Просмотр списка настольных игр, игротек, мест
                        проведения, организаторов, пользователей
                    \item Изменение информации о настольных играх, \mbox{местах}
                        проведения, организаторах, игротеках
                          \item Одобрение или отклонение заявок от
                              \mbox{организаторов}
                          \item Изменение прав доступа пользователей
                          \item Удаление пользователей
                      \end{itemize}
                  \end{minipage}\\
            \hline
        \end{tabular}
    \end{threeparttable}
    \end{center}
\end{table} 

Диаграммы вариантов использования для Гостя, Игрока, Организатора и
Администратора представлены на
рисунках~\ref{img:Guest},~\ref{img:Player},~\ref{img:Organizer},~\ref{img:Admin}
соответственно.

\imgh{Guest}{h!}{4.5cm}{Use-Case диаграмма для Гостя}
\imgh{Player}{h!}{7cm}{Use-Case диаграмма для Игрока}
\imgh{Organizer}{h!}{9cm}{Use-Case диаграмма для Организатора}
\clearpage
\imgh{Admin}{h!}{9cm}{Use-Case диаграмма для Администратора}

\section{\label{head:01}Формализация данных}

На основании сформулированных требований база данных должна содержать информацию
о:

\begin{itemize}
    \item игротеках;
    \item настольных играх, по которым проводятся игротеки;
    \item организаторах;
    \item местах проведения;
    \item игроках;
    \item пользователях.
\end{itemize}

Последние четыре сущности характерны для многих предметных областей, поэтому для
них приведено только краткое описание (таблица~\ref{tab:03}), а подробный анализ
приводится для первых двух сущностей, являющихся основными в рассматриваемой
предметной области (их краткое описание также представлено в
таблице~\ref{tab:03}).

\clearpage
\begin{table}[h!]
    \begin{center}
    \begin{threeparttable}
        \captionsetup{format=hang,justification=raggedright,
                      singlelinecheck=off}
        \caption{\label{tab:03}Категории данных и сведения о них}
        \renewcommand{\arraystretch}{1.5}
        \begin{tabular}{|p{4cm}|p{12cm}|}
            \hline
            \bfseries Категория & \bfseries Сведения\\
            \hline
            \begin{minipage}[t]{\linewidth}
              \begin{multicols}{1}
                \begin{itemize}[leftmargin=0mm,labelsep=0mm,nosep,after=\strut]
                  \item[] Игротека
                  \item[]
                  \item[]
                  \item[]
                  \item[]
                  \item[]
                  \item[]
                  \item[]
                  \item[]
                \end{itemize}
              \end{multicols}
            \end{minipage}
          & \begin{minipage}[t]{\linewidth}
              \begin{multicols}{2}
                \begin{itemize}[leftmargin=0mm,labelsep=0mm,nosep,after=\strut]
                  \item[] ID
                  \item[] Название
                  \item[] Организатор
                  \item[] Время начала
                  \item[] Дата проведения
                  \item[] Возможность покупки игр
                  \item[] Место проведения
                  \item[] Продолжительность
                  \item[] Стоимость участия
                  \item[] Время регистрации
                \end{itemize}
              \end{multicols}
            \end{minipage}\\
            \hline
            \begin{minipage}[t]{\linewidth}
              \begin{multicols}{1}
                \begin{itemize}[leftmargin=0mm,labelsep=0mm,nosep,after=\strut]
                  \item[] Организатор
                  \item[]
                  \item[]
                  \item[]
                  \item[]
                \end{itemize}
              \end{multicols}
            \end{minipage}
          & \begin{minipage}[t]{\linewidth}
              \begin{multicols}{2}
                \begin{itemize}[leftmargin=0mm,labelsep=0mm,nosep,after=\strut]
                  \item[] ID
                  \item[] Название
                  \item[] Адрес
                  \item[] Сайт
                  \item[] Email
                  \item[] Телефон
                \end{itemize}
              \end{multicols}
            \end{minipage}\\
            \hline
            \begin{minipage}[t]{\linewidth}
              \begin{multicols}{1}
                \begin{itemize}[leftmargin=0mm,labelsep=0mm,nosep,after=\strut]
                  \item[] Место 
                  \item[] проведения
                  \item[]
                  \item[]
                  \item[]
                  \item[]
                  \item[]
                \end{itemize}
              \end{multicols}
            \end{minipage}
          & \begin{minipage}[t]{\linewidth}
              \begin{multicols}{2}
                \begin{itemize}[leftmargin=0mm,labelsep=0mm,nosep,after=\strut]
                  \item[] ID
                  \item[] Адрес
                  \item[] Сайт
                  \item[] Название
                  \item[] Тип
                  \item[] Email
                  \item[] Телефон
                \end{itemize}
              \end{multicols}
            \end{minipage}\\
            \hline
            \begin{minipage}[t]{\linewidth}
              \begin{multicols}{1}
                \begin{itemize}[leftmargin=0mm,labelsep=0mm,nosep,after=\strut]
                  \item[] Настольная
                  \item[] игра
                  \item[]
                  \item[]
                  \item[]
                  \item[]
                  \item[]
                  \item[]
                  \item[]
                \end{itemize}
              \end{multicols}
            \end{minipage}
          & \begin{minipage}[t]{\linewidth}
              \begin{multicols}{2}
                \begin{itemize}[leftmargin=0mm,labelsep=0mm,nosep,after=\strut]
                  \item[] ID
                  \item[] Название
                  \item[] Производитель
                  \item[] Год выпуска
                  \item[] Мин. число игроков
                  \item[] Макс. число игроков
                  \item[] Мин. возраст
                  \item[] Макс. возраст
                  \item[] Мин. время игры
                  \item[] Макс. время игры
                \end{itemize}
              \end{multicols}
            \end{minipage}\\
            \hline
            \begin{minipage}[t]{\linewidth}
              \begin{multicols}{1}
                \begin{itemize}[leftmargin=0mm,labelsep=0mm,nosep,after=\strut]
                  \item[] Игрок
                  \item[]
                  \item[]
                \end{itemize}
              \end{multicols}
            \end{minipage}
          & \begin{minipage}[t]{\linewidth}
              \begin{multicols}{2}
                \begin{itemize}[leftmargin=0mm,labelsep=0mm,nosep,after=\strut]
                  \item[] ID
                  \item[] Имя
                  \item[] Лига
                  \item[] Рейтинг
                \end{itemize}
              \end{multicols}
            \end{minipage}\\
            \hline
            \begin{minipage}[t]{\linewidth}
              \begin{multicols}{0}
                \begin{itemize}[leftmargin=0mm,labelsep=0mm,nosep,after=\strut]
                  \item[] Пользователь
                  \item[]
                  \item[]
                  \item[]
                \end{itemize}
              \end{multicols}
            \end{minipage}
          & \begin{minipage}[t]{\linewidth}
              \begin{multicols}{2}
                \begin{itemize}[leftmargin=0mm,labelsep=0mm,nosep,after=\strut]
                  \item[] ID
                  \item[] Роль
                  \item[] Логин
                  \item[] Пароль
                \end{itemize}
              \end{multicols}
            \end{minipage}\\
           \hline
        \end{tabular}
    \end{threeparttable}
    \end{center}
\end{table} 
\vspace{-0.5cm}

На основе анализа существующих решений можно выявить основные сведения о
настольных играх и игротеках, которые используются при их описании. Так, для
описания настольной игры используют следующие характеристики:

\begin{itemize}
    \item название;
    \item производитель;
    \item год выпуска;
    \item возможное количество игроков в формате <<от ... до ...>>;
    \item время игры также в формате <<от ... до ...>>;
    \item рекомендованный возраст участников в формате возрастных категорий
        (<<6+>>, <<12+>>, <<18+>> и т.~п.) или диапазона возрастов
        (<<от~...~до~...>>);
\end{itemize}

% Также возможно использование следующих характеристик:
% \begin{itemize}
%     \item вес;
%     \item комплектация;
%     \item правила в виде отдельного файла;
%     \item описание игры;
%     \item изображения коробки и комплектующих.
% \end{itemize}

%Так как по сформулированным требованиям настольные игры используются для поиска
%необходимых игротек, последние описанные характеристики в сущность настольной
%игры включены не будут.

Для описания игротеки в рассмотренных решениях используются:
\begin{itemize}
    \item название;
    \item адрес места проведения;
    \item время начала и продолжительность;
    \item стоимость участия;
    \item ссылка на регистрацию.
\end{itemize}

Так как регистрация на игротеку будет происходить в самой системе, ссылка на
регистрацию, как сведение об игротеке храниться не будет, но в соответствии с
сформулированными требованиями будет добавлено время, за которое можно
зарегистрироваться на игротеку.

Связи между описанными сущностями представлены на ER-диаграмме
(рисунок~\ref{img:BG-ER}).

\imgh{BG-ER}{ht!}{16.7cm}{ER-диаграмма}

\section{Выбор модели базы данных}

На основании сформулированных требований система должна включать базу данных для
хранения информации обо всех сущностях, описанных в подразделе~\ref{head:01}.

База данных (БД) --- это совокупность данных, организованных по определенным
правилам, предусматривающим общие принципы описания, хранения и манипулирования
данными, независимая от прикладных программ~\cite{gost01}.

Основой любой базы данных является модель данных --- формализованное описание
структур единиц информации и операций над ними в информационной системе.
Модель данных определяет логическую структуру базы данных, то есть
способ хранения, организации и обработки информации~\cite{book01}.

Существуют различные модели данных, каждая из которых имеет свои достоинства и
недостатки, поэтому необходимо провести анализ моделей и выбрать модель базы
данных, соответствующая модель данных которой наиболее полно удовлетворяет
требованиям реализуемой системы.

Модели баз данных можно разделить на три основные категории: дореляционные,
реляционные и постреляционные.

\subsection{Дореляционные базы данных}

К дореляционным моделям~\cite{book03} относят модели, предшедствующие
реляционной: иерархическую, сетевую и модель на основе инвертированных
списков.

\textbf{Иерархическая модель}  --- представление базы данных в виде древовидной
структуры, состоящей из объектов разных уровней, где связи между записями
выражаются в виде отношений предок/потомок, а у каждой записи есть ровно одна
родительская запись.

Достоинствами иерархической модели являются:
\begin{itemize}
    \item простота модели (схожая, например, со структурой компании или
        генеалогическим деревом);
    \item быстродействие за счет реализации отношения предок/потомок в виде
        физических указателей.
\end{itemize}

Недостатками иерархической модели являются:
\begin{itemize}
    \item необходимость дублирования деревьев для связи многие-ко-многим;
    \item отсутствие ссылок между записями, не входящими в одну иерархию и, как
        следствие, невозможность хранения в базе данных порожденного узла без
        соответсвующего исходного.
\end{itemize}

Для устранения недостатков иехрархической модели была разработана
\textbf{сетевая модель}, в которой одна запись может участвовать в нескольких
отношениях предок/потомок.

К достоинствам сетевой модели можно отнести:
\begin{itemize}
    \item уменьшение дублирования данных по сравнению с иерархической моделью;
    \item гибкость за счет возможности создания дополнительных связей.
\end{itemize}

Недостатками сетевой модели являются:
\begin{itemize}
    \item сложность и запутанной структуры;
    \item ослабление контроля целостности вследствие допустимости установления
        произвольных связей между записями.
\end{itemize}

В силу сложности практического использования иерархической и сетевой модели
появилась \textbf{модель на основе инвертированных списках}. Организация доступа
к данным на основе инверитированных списков используется в современных
реляционных системах, однако в них пользователи не имеют непосредственного
доступа к инвертированным спискам.

Недостатками такой модели являются:
\begin{itemize}
    \item отсутствие строгого математического аппарата;
    \item отсутствие средств для описания ограничений целостности базы данных;
    \item большая трудоемкость программирования запросов к базе данных.
\end{itemize}

Общим недостатком дореляционных баз данных является поддержка доступа к данным
только путем создания прикладных программ и, как следствие, оптимизация доступа
к данным пользователем без какой-либо поддержки системы.

\subsection{Реляционные базы данных}

Реляционные базы данных основаны на реляционной модели, представляющей сущности
и связи между ними представлены в виде отношений --- двумерных таблиц, каждая
строка (кортеж) является записью, содержащей данные о конкретном объекте данной
сущности, а столбцы ее свойствам, или атрибутам. То есть реляционная база данных
--- это совокупность отношений, содержащих всю информацию, которая должна
храниться в базе данных.

В каждом отношении выделяют первичный ключ --- атрибут или набор атрибутов,
однозначно идентифицирующий каждый из кортежей отношения. В этом же или другом
отношении может быть создан столбец (их набор), ссылающийся на первичный ключ
--- внешний ключ, с помощью которого реализуются связи между таблицами базы
данных~\cite{book03}.

При этом для поддержания целостности данных в реляционных БД соблюдаются
требованиям ACID:
\begin{itemize}
    \item атомарность (atomicity) --- для каждой транзакции либо выполняются все
        операции внутри нее, либо не выполняется ни одной, то есть
        транзакции являются атомарными и работает принцип <<все или ничего>>;
    \item согласованность (consistency) --- выполнение транзакции не может
        перевести систему в несогласованное состояние, то есть база данных
        всегда остается в согласованном состоянии;
    \item изолированность (isolation) --- на результат транзакции не влияют
        другие транзакции, которые происходят параллельно с ней;
    \item долговечность (durabiliry) --- любые изменения сохраняются в базе
        данных несмотря на сбои и действия пользователей.
\end{itemize}

Реляционные базы данных поддерживают SQL --- язык структурированных запросов для
определения и обработки данных. SQL является одним из наиболее гибких и
стандартизированных языков запросов, что позволяет минимизировать ряд рисков
для разработчиков~\cite{art04}.

Достоинствами реляционных баз данных являются:
\begin{itemize}
    \item единый способ представления сущностей и связей между ними:
        все --- отношение;
    \item соответствие требованиям ACID;
    \item использование стандартизированного языка запросов SQL.
\end{itemize}

Недостатками реляционных баз данных являются:
\begin{itemize}
    \item вертикальная масштабируемость;
    \item отсутствие возможности представить некоторые данные в виде отношения;
    \item увеличение времени работы при увеличении числа отношений (например,
        при нормализации);
    \item трудозатраты на проектирование базы данных.
\end{itemize}

\subsection{Постреляционные базы данных}

Постреляционные базы данных --- базы данных с динамическим представлением
структуры данных~\cite{art04}. В таких базах данных используются гибкие модели
данных, которые определяют следующие типы баз данных~\cite{art05}:
\begin{itemize}
    \item колоночные БД хранят данные по столбцам, число которых от строки к
        строке может изменяться;
    \item хранилища <<ключ-значение>> для хранения объектов используют
        хеш-таблицу, в которой находятся пары из уникального ключа и
        указателя на конкретный объект данных;
    \item документноориентированные БД хранят данные в виде коллекций
        документов (например, в форматах JSON, XML и~др.), причем каждая
        запись может содержать различные по количеству и типам данные.
    \item графовые БД используются для хранения данных с большим числом
        связей и представляют элементы базы данных в виде вершин графа, а
        отношения между ними --- в виде ребер между соответствующими
        вершинами.
\end{itemize}

Достоинствами постреляционных баз данных являются:
\begin{itemize}
    \item горизонтальная масштабируемость;
    \item гибкость моделей данных;
    \item возможность хранения неструктурированной информации.
\end{itemize}

Недостатками постреляционных баз данных являются:
\begin{itemize}
    \item смягчение требований ACID;
    \item привязанность приложения к конкретной СУБД в силу собственных
        языков запросов (отсутствие поддержки SQL).
\end{itemize}

\subsection*{Вывод}

В разрабатываемой базе:

\begin{itemize}
    \item данные являются структурированными, причем их структура не подвержена
        частым изменениям;
    \item необходима поддержка целостности данных при наличии нескольких
        пользователей, то есть соответствие требованиям к транзакционным
        системам;
    \item необходимо выполнение сложных запросов.
\end{itemize}

Таким образом, в соответствии с перечисленными требованиями к базе данных и
свойствами основных категорий моделей данных необходимо выбрать реляционную
базу данных.

\section*{Вывод}

В данном разделе был проведен анализ предметной области настольных игр и игротек
с помощью исследования существующих решений. На основе анализа предметной
области была формализована задача и данные и описаны типы пользователей. Также
по результатам сравнения моделей баз данных для решения задачи была выбрана
реляционная модель данных.
