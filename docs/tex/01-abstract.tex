\begin{essay}{}
    Ключевые слова: база данных, система управления базами данных, PostgeSQL,
    настольные игры, игротеки.

    В данной работе проведено проектирование и разработка базы данных настольных
    игр и игротек. 

    В разделе~\ref{analys} проведен анализ существующих решений, выявлен их
    основной недостаток: отсутствие возможности для игроков регистрироваться на
    игротеки разных организаторов через единую систему. Также проведена
    формализация задачи, данных и типов пользователей. В качестве модели базы
    данных выбрана релияционная.

    В разделе~\ref{design} проведено проектирование базы данных: необходимые
    таблицы, хранимые процедуры и функции, триггеры и роли, --- а также проведно
    проектирование приложения.

    В разделе~\ref{impl} в качестве используемой СУБД выбрана PostgreSQL, а в
    качестве средств реалзиации выбраны: C\#, Entity Framework, Blazor и Visual
    Studio. Также приведены детали реализации и примеры работы программы.

    В разделе~\ref{research} проведено исследование зависимости времени
    выполнения запроса от индексации при различном количестве записей.

    К разделам сделаны выводы, в заключении подведены итоги.
\end{essay}
