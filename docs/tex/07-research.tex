\chapter{\label{research}Исследовательская часть}

\section{Технические характеристики}

Технические характеристики устройства, на котором выполнялось тестирование:

\begin{itemize}
	\item Операционная система: Windows 10 Home.
	\item Память: 8 GiB.
    \item Процессор: Intel® Core™ i5-8265U, 4 физических ядра, 8 логических
        ядра.
\end{itemize}

Эксперимент проводился на ноутбуке, включенном в сеть электропитания. Во
время тестирования ноутбук был нагружен только встроенными приложениями
окружения, окружением, а также непосредственно системой тестирования.

\section{Описание эксперимента}

Целью эксперимента является проверка зависимости времени выполнения запроса от
индексации на различных количествах записей в таблице.

Для проведения эксперимента выбрана таблица игротек как реализация основной
сущности в системе. Для индексации выбрано поле даты как наиболее
востребованное при поиске и сортировке. Тестирование проводилось на времени
выполнения запроса сортировки при количествах записей от 10 до 1000.

Создание индекса приведено на листинге~\ref{lst:index}.

{
\captionsetup{format=hang,justification=raggedright,
              singlelinecheck=off,width=16cm}
\listingfile{sql}{index.sql}{}{Создание таблицы мест проведения}{index}
}

В ходе эксперимента время выполнения запросы измерялось с помощью встроенных
инструментов СУБД PostgreSQL~(листинг~\ref{lst:query}).

{
\captionsetup{format=hang,justification=raggedright,
              singlelinecheck=off,width=16cm}
\listingfile{sql}{query.sql}{}{Создание таблицы мест проведения}{query}
}

Примеры выполнения данного запроса с индексацией и без нее представлены на
рисунках~\ref{img:indexQuery},~\ref{img:seqQuery} соответственно.

\imgh{indexQuery}{h!}{1.5cm}{План запроса на сортировку по дате с
индексированием}
\imgh{seqQuery}{h!}{2.5cm}{План запроса на сортировку по дате без индексирования}

\section{Результат эксперимента}

В таблице~\ref{tab:03} приведены результаты измерений. График, иллюстрирующий
данную таблицу приведен на рисунке~\ref{img:ResearchGraphic}.

\begin{table}[ht!]
\captionsetup{format=hang,justification=raggedright,
              singlelinecheck=off,width=12cm}
\centering
\caption{Сравнение СУБД\label{tab:02}}
\begin{tabular}[Hc]{|p{3cm}|p{3cm}|p{3cm}|}
    \hline
    \multicolumn{1}{|c}{\textbf{Число записей}} &
    \multicolumn{1}{|c|}{\textbf{Без индекса, мс}} &
    \multicolumn{1}{c|}{\textbf{С индексом, мс}}\\
    \hline
    10   & 0.062 & 0.035 \\
    \hline
    30   & 0.100 & 0.056 \\
    \hline
    60   & 0.127 & 0.113 \\
    \hline

    100  & 0.195 & 0.116 \\
    \hline
    300  & 0.455 & 0.311 \\
    \hline
    600  & 0.867 & 0.628 \\
    \hline
    1000 & 1.378 & 1.029 \\
    \hline
\end{tabular}
\end{table}

\imgh{ResearchGraphic}{h!}{7cm}{График зависимости времени выполнения запроса
от числа записей в таблице без и с индексированием}

\section*{Вывод}

Из полученных результатов можно сделать вывод, что использование индексирования
в среднем дает уменьшение времени выполнения запроса в 1.5 раза.
