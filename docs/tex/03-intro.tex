\chapter*{ВВЕДЕНИЕ}
\addcontentsline{toc}{chapter}{ВВЕДЕНИЕ}

Настольные игры --- это игры, основанные на манипуляции инвентарем (карточками,
кубиками, фишками и т.~п.), размещаемым на обычном или специально сделанном
столе~\cite{art01}.

На сегодняшний день рынок настольных игр представляет собой одну из наиболее
молодых и стремительно развивающихся областей мировой
экономики~\cite{art02}. Это связно с тем, что несмотря на современные условия
информатизации настольные игры набирают популярность и продолжают сохранять
востребованность в широкой среде потребителей.

Отечественный рынок настольных игр молод, но стремительно усиливает свои
позиции: за счет сегментации рынка по возрасту и категориям (семейные,
стратигические, для вечеринок и др.) настольные игры отделились от рынка
детских товаров и стали самостоятельным сектором экономики.

В отличие от других рынков прямая реклама настольных игр работает не эффективно.
При этом наилучший способ прорекламировать игру --- посадить человека играть.
Поэтому в сфере настольных игры приобрели популярность игротеки --- мероприятия,
на котором гости могут поиграть в различные игры, завести знакомства, опробовать
новинки или ознакомиться с ассортиментом~\cite{art03}.

В силу молодости рынка программные решения для анонсов игротек, регистрации на
них, получения информации об проводимых играх, как будет показано в одном из
разделов данной работы, по большей части представлены в виде интернет-магазинов,
в которых информация об игротеках либо представляется текстовым списком на
отдельной странице, либо появляется в новостном разделе, либо вовсе отсутствует.
Поэтому поиск необходимых игротек и регистрация на них становятся
проблематичными.

Таким образом, \textbf{целью данной работы} является проектирование и реализация
базы данных настольных игр и игротек, а также разработка приложения доступа к
базе данных с возможностью просмотра, добавления, удаления и редактирования
информации о настольных играх и игротеках, регистрации на игротеки и составления
списка предпочитаемых игр.

Для достижения поставленной цели необходимо решить следующие \textbf{задачи}:
\begin{itemize}[left=\parindent]
    \item провести анализ предметной области;
    \item формализовать задачу;
    \item провести анализ баз данных и систем управления базами данных;
    \item спроектировать базу данных и архитектуру приложения;
    \item реализовать базу данных и приложение для доступа к ней.
\end{itemize}

